
\subsection{Motivation}

Collectively, we observed significant topological modification of 
the human brain functional connectome
across the lifespan, while the general structure of the connectomics 
was stable over time.  
These results may be
relevant for understanding the changes in neural circuits
that underlie age-related variation in cognition and behavior across 
the lifespan. 
\cite{Cao2014}

There are several potential improvements and extensions to graph attention networks that could be
addressed as future work, such as overcoming the practical problems described in subsection 2.2 to
be able to handle larger batch sizes. A particularly interesting research direction would be taking
advantage of the attention mechanism to perform a thorough analysis on the model interpretability.
Moreover, extending the method to perform graph classification instead of node classification would
also be relevant from the application perspective. Finally, extending the model to incorporate edge
features (possibly indicating relationship among nodes) would allow us to tackle a larger variety of
problems.



\cite{Velickovic2018}

The first one represents the time axis where we can
track the brain connectivity changes over time from birth to
age or foresee a transition from a healthy to a disordered
state.
By deepening our understanding of the relationship
between the brain connectivities at a first timepoint and its
follow-up acquisitions, we may produce novel diagnostic
tools for better identifying neurological disorders at a very
early stage (e.g., Alzheimer’s disease and autism) [12], [13]

the third axis refers to the domain in
which the brain data was collected, which is commonly
referred to in network neuroscience as the ‘neuroimaging
modality’ (e.g., functional MRI or diffusion MRI) utilized
to generate the brain connectome type (e.g., functional or
structural). Such cross-domain multimodal representation of
the brain connectivity provides invaluable complementary
information for brain mapping in health and disease.
\cite{Bessadok2022}

The human cerebral cortex is organized into functionally segregated 
neuronal populations connected by the anatomical pathways. 
White matter fiber tracts form a connectome of structural 
connectivity at the macroscale. This structural connectome exhibits 
a complex network topology characterized by non-random properties, 
including small-world architecture2, segregated 
communities3, and a core of densely inter-connected hubs4. 
These topological patterns support the communication dynamics on 
structural networks and coordinate the temporal synchronization of 
neural activity—termed functional connectivity—between cortical regions5-8.

Understanding how structural connectivity shapes functional connectivity 
patterns is central to neuroscience. 

higher structure-function coupling has been related to better performance 
in executive function10,22, and abnormal patterns of the
coupling are associated with a wide range of psychiatric and 
neurological disorders, such as major depressive disorder26, 
bipolar disorder27, attention deficit hyperactivity disorder28, 
and Parkinson’s disease29.

\cite{Chen2024}


Brain network analysis has been demonstrated to be effective for the 
diagnosis and prognosis of neurological disorders (Cao et al. 2015; Tadayonnejad
and Ajilore 2014; Wang et al. 2017). \cite{Zhang2022}

\subsection{ Dataset Challenges}
Another challenge of the current brain network
dataset is data insufficiency, which will further 
limit the progress of big data mining on brain network studies. 
For example, the current brain network datasets may not be easy 
to utilize for the group difference studies based on the deep 
learning model since the number of networks in a few subgroups 
may not be enough to train the neural networks.
Instead of enlarging the current dataset, technical methods 
in addressing data quantity issues should also be strictly 
considered. These methods include but are not limited to 
data augmentation techniques, fast algorithms for brain 
network constructions from neuroimaging data, multisite 
learning for dataset combinations, and pre-trained model
development. \cite{Tang2023}


