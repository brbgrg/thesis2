\subsection{Techniques used}


\subsection{Age-Related Changes in Brain Networks}

Collectively, we observed significant topological modification of 
the human brain functional connectome
across the lifespan, while the general structure of the connectomics 
was stable over time. These results may be
relevant for understanding the changes in neural circuits
that underlie age-related variation in cognition and behavior across 
the lifespan. 
\cite{Cao2014}


Human brain function undergoes complex transformations across the lifespan. We
employed resting-state functional MRI and graph-theory approaches to systematically
chart the lifespan trajectory of the topological organization of human whole-brain 
functional networks in 126 healthy individuals ranging in age from 7 to 85 years. Brain
networks were constructed by computing Pearson’s correlations in blood-oxygenation
level-dependent temporal fluctuations among 1024 parcellation units followed by
graph-based network analyses. We observed that the human brain functional connectome
exhibited highly preserved non-random modular and rich club organization over the entire
age range studied. Further quantitative analyses revealed linear decreases in modularity
and inverted-U shaped trajectories of local efficiency and rich club architecture. Regionally
heterogeneous age effects were mainly located in several hubs (e.g., default network, dorsal
attention regions). Finally, we observed inverse trajectories of long- and short-distance
functional connections, indicating that the reorganization of connectivity concentrates and
distributes the brain’s functional networks. Our results demonstrate topological changes in
the whole-brain functional connectome across nearly the entire human lifespan, providing
insights into the neural substrates underlying individual variations in behavior and cog
nition. These results have important implications for disease connectomics because they
provide a baseline for evaluating network impairments in age-related neuropsychiatric
disorders.
\cite{Cao2014}



- decreased FC-SC coupling with aging 
- increased FC variability
- increased inter-subject variability in both SC and FC networks
- decreased modular segregation in FC networks

\subsubsection {Age-Related Changes in Structure-Function Coupling}
Older subjects may exhibit a weaker coupling between SC and FC networks. 
In contrast, young and adult subjects might show stronger SC-FC coupling, 
where structural modules are more predictive of functional modules.

Structural connections may degrade over time, while functional connections maintain 
higher variability, leading to a decoupling of SC and FC modules.

\subsubsection {Modular Reconfiguration with Aging}

Modular consistency in SC networks might decline with age.
Age-related atrophy can lead to disruptions in connectivity for older adults.

The functional networks of older individuals are likely to show greater 
reconfiguration over time. FC becomes more integrated to compensate for the structural decline.


\subsubsection {Variability Across Subjects in Older Adults}
The inter-subject variability in both SC and FC networks is expected to increase with age. 
Older individuals may show more subject-specific modular patterns, as both SC and FC networks 
are likely to deviate more from the typical patterns seen in younger individuals or adults.

\subsubsection{Modular Integration vs Segregation}
In younger individuals, SC and FC networks tend to have more modular segregation, 
meaning distinct brain regions form highly specialized communities. 

In adults, there may be a balance between integration and segregation. 
SC modules may remain stable, while FC modules start to show increased integration.

In older individuals, modular integration increases. FC modules become less segregated, 
indicating more global communication across brain regions. 
This is often observed as a compensatory mechanism to maintain cognitive function as 
structural connections degrade.


\cite{Puxeddu2022}
