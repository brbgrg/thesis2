\subsection{Temporal and Spatial Resolution Limitations}
The proposed method could take advantage of the high spatial 
resolution of fMRI and high temporal resolution of
MEG to achieve the highest sex classifcation performance of 85.2%.
\cite{Zhao2022}

The Schaefer 100 parcellation scheme is generally considered a low-resolution 
scheme. In the context of brain parcellations, “low-resolution” typically 
refers to schemes with fewer regions or nodes, which provide a broader 
overview of brain connectivity and function. Higher-resolution schemes, 
like those with 300, 500, or 1000 nodes, offer more detailed and granular 
insights.

Diffusion magnetic resonance imaging (dMRI) and functional MRI reveals the large
scale SC across different brain regions. Electrophysiological methods (i.e.
MEG/EEG) provide direct measures of neural activity and exhibits complex 
neurobiological temporal dynamics which could not be solved by
fMRI. However, most of existing multimodal analytical methods collapse the brain 
measurements either in space or time domain and fail
to capture the spatio-temporal circuit dynamics. In this paper, we propose a novel 
spatio-temporal graph Transformer model to integrate the
structural and functional connectivity in both spatial and temporal domain.

The proposed method is evaluated with the meta-analysis to explore the
behavioral relevance of different brain regions and characterize the brain
dynamical organization into low level functions region (i.e. sensory) and the
complex function regions (i.e. memory).
\cite{Zhao2022a}

\subsection{Pipeline limitations}
While effective, this model is not trained in an end-to-end
fashion since the clustering, cluster-specific CBT generation, 
and the population CBT estimation blocks are learned
disjointly. Therefore, cumulated errors across these blocks
might produce a less centered brain template
\cite{Bessadok2022}

\subsection{Interpretability Limitations}
To highlight the importance of the brain ROIs, we introduce the explainable
causal representation to encourage the reasonable node selection process. We
train an explanation model to explain the multmodal graph representation approach based on granger causality.
\cite{Zhao2022}

\subsection{Instrumentation Limitations}
Several issues should be considered in interpreting the
current findings. First, we regressed out the global signal to partly reduce 
physiological and other global noise.
We also repeated analyses without global signal regression and found that 
most of our findings were conserved,
and the modularity failed to detect age effects. This finding might be an 
indication of the fact that global signal
removing reduces the effects of physiological and other
noise across the whole brain and makes the metrics more
comparable across participants. Second, to mitigate the
effects of in-scanner head motion, we used the Friston24 regression model in 
the preprocessing and added a
motion-related parameter, meanFD, in group level, which
has been proven to be a promising way to reduce the impact
of motion artifacts on both individual and group-level
outcomes (Yan et al., 2013). However, the effects of residual
motion may remain in our results. Specifically, head movement has been found 
to have a distinct impact on long- and short-connections, and it significantly correlates with age
(Mowinckel et al., 2012; Power et al., 2012; Satterthwaite
et al., 2012; Van Dijk et al., 2012), implying a complex
role of head motion in the changes of distance-dependent
connectivity. This important issue should be studied carefully in future.

Third, it is challenging to map the brain’s
parcellation-based functional connectome appropriately
and precisely (Butts, 2009; Smith et al., 2011). We used a
random-generated high-resolution template and showed
the repeatability of most of our findings across different parcellation schemes. 
Notably, non-uniform findings
between templates were also observed, which may originate from the 
differences in brain units. These findings
indicate that the age effects on functional networks
were template dependent. Nevertheless, the development
and adoption of novel network tools and whole-brain
connectivity-based parcellation approaches in the future
will provide additional insight into the age effects on the
functional connectome

Fourth, we attempted to explore
the age-related differences of brain functional networks
over a continuous age range that covered both development and aging. 
However, the analyzed samples were
not perfectly distributed across the entire lifespan. The
number of young adults was greater than the number of
older people. We employed linear and quadratic (nonlinear) models to explore 
the age changes across the human
lifespan. The incomplete distribution of ages in our sample may have 
affected parametric curve fitting. Exploration
of larger R-fMRI datasets using non-parametrical models
(e.g., smoothing splines) may reveal more robust and complex maturational 
processes (Fjell and Walhovd, 2010).
Finally, the age-related functional connectome changes
were detected based on the cross-sectional data and thus
could be potentially influenced by unbalanced cohort distributions. 
Because different age cohorts may be different
in substantive ways, investigations of the longitudinal network dynamics 
should be taken in the future to reveal
the nature of age-related changes (Fjell and Walhovd,2010)

\cite{Cao2014}

