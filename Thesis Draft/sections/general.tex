improve comparison of S-F coupling 
across age groups
using GNN for community detection (RL)




Graph classification (disease diagnosis)
attention-based network

1. combine the structural and functional brain graphs into a single 
graph representation:
learn the embedding of a single graph that is represented by
an adjacency matrix denoting the structural connectivities
and a feature matrix denoting the functional connectivities

2.  resulting learned latent representation was passed to
a multi-layer perceptron classifier to predict frontal lobe
epilepsy, temporal lobe epilepsy, and healthy subjects.

3. add interpretation of the captured features in the embedding space 
which can help better understand the original connectivity pattern that 
yields the illness
(Edge-Weighted GAT layer, followed by Diffpool layers)



Graph integraton
DGN (deep graph normalizer)
a GNN-based model that integrates a population of multiview brain
graphs named “multigraph” into a connectional brain template (CBT) 
in an end-to-end learning fashion

Specifically, a multigraph encodes different facets of the brain where
the exisiting connection between two nodes is encoded in
a set of edges of multiple types. Each edge type denotes a
particular form of brain connectivity derived from a particular neuroimage data. 

DGN was proposed to integrate the complementary information across all subjects of 
the population by 
(i) training a set of edge conditioned convolutional
layers each learns the embeddings of brain regions in the
graph, 
(ii) introduce a series of tensor operations to calculate
the pairwise absolute difference of each pair of the learned
node embeddings which results in a subject-based CBT, and
finally, 
(iii) estimate the CBT of the population by selecting
the median of all subject-based CBTs