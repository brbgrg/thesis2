\subsection{Deep brain network representation learning}
With the development of artificial intelligence (AI) techniques,
learning-based methods (e.g., machine learning, deep learning) are
broadly investigated and applied to brain network data for different
research purposes. Most of these learning methods are based on the graph
neural networks (GNN), a class of deep neural networks for graph structured 
data representations.27–29,124–128 Many research objectives
on brain network learning have been proposed in recent years. For
example, a few studies focus on developing deep learning methods to
model the multiview representations across different modalities-derived
brain network data. Some other studies focus on investigating the
interpretability of the deep learning models to yield biological insights
(e.g., finding new biomarkers that closely relate brain networks to
different neurological disorders) for the model outcomes. As shown in
Fig. 1, we summarize these studies based on these research objectives. \cite{Tang2023}