NetworkX: 



\subsection{BrainNet Viewer: network visualization techniques}
(e.g., BrainNet visualization as 
demonstrated in \cite{Xia2013}) are sometimes employed to visually
represent distinctions within brain networks \cite{Tang2023}

Intriguingly, the normalized rich club coefficient ˚norm showed inverted
U shaped lifespan trajectories (p = 0.01, r2 = 0.10; Fig. 4B),
indicating that the brain’s functional rich club architecture increased 
until approximately 40 years of age and
decreased at older ages. These findings persisted over a
range of hub thresholds (Figs. 4A and S3). The 3D surface
visualizations of the results were implemented using the
Brain Net Viewer (www.nitrc.org/projects/bnv) (Xia et al.,
2013).
\cite{Cao2014}


Keiriz (2018) - Paper describing NeuroCave and related visualization work.
LaPlante (2014) - The Connectome Visualization Utility, which is implemented in Python; repo here.
Margulies (2013) - A paper on connectome viz, highlighting more examples and resources.
NeuroMArVL - Makes network pictures on a brain surface, in a circle diagram, or as a spring-embedding (2D or 3D).
Nilearn - Plenty of brain visualization inspiration to be had here.
Xia (2013) - BrainNet Viewer, which can also be found on NITRC.
https://github.com/faskowit/brain-networks-across-the-web


